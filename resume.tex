%-------------------------
% Resume in Latex
% Author : Dylan Chen
% Based off of: https://github.com/sb2nov/resume
% License : MIT
%------------------------

\documentclass[letterpaper,11pt]{article}

\usepackage{latexsym}
\usepackage[empty]{fullpage}
\usepackage{titlesec}
\usepackage{marvosym}
\usepackage[usenames,dvipsnames]{color}
\usepackage{verbatim}
\usepackage{enumitem}
\usepackage[hidelinks]{hyperref}
\usepackage{fancyhdr}
\usepackage[english]{babel}
\usepackage{tabularx}
\input{glyphtounicode}


%----------FONT OPTIONS----------
% sans-serif
% \usepackage[sfdefault]{FiraSans}
% \usepackage[sfdefault]{roboto}
% \usepackage[sfdefault]{noto-sans}
% \usepackage[default]{sourcesanspro}

% serif
% \usepackage{CormorantGaramond}
% \usepackage{charter}


\pagestyle{fancy}
\fancyhf{} % clear all header and footer fields
\fancyfoot{}
\renewcommand{\headrulewidth}{0pt}
\renewcommand{\footrulewidth}{0pt}

% Adjust margins
\addtolength{\oddsidemargin}{-0.5in}
\addtolength{\evensidemargin}{-0.5in}
\addtolength{\textwidth}{1in}
\addtolength{\topmargin}{-.5in}
\addtolength{\textheight}{1.0in}

\urlstyle{same}

\raggedbottom
\raggedright
\setlength{\tabcolsep}{0in}

% Sections formatting
\titleformat{\section}{
  \vspace{-4pt}\scshape\raggedright\large
}{}{0em}{}[\color{black}\titlerule \vspace{-5pt}]

% Ensure that generate pdf is machine readable/ATS parsable
\pdfgentounicode=1

%-------------------------
% Custom commands
\newcommand{\resumeItem}[1]{
  \item\small{
    {#1 \vspace{-2pt}}
  }
}

\newcommand{\resumeSubheading}[4]{
  \vspace{-2pt}\item
    \begin{tabular*}{0.97\textwidth}[t]{l@{\extracolsep{\fill}}r}
      \textbf{#1} & #2 \\
      \textit{\small#3} & \textit{\small #4} \\
    \end{tabular*}\vspace{-7pt}
}

\newcommand{\resumeSubSubheading}[2]{
    \item
    \begin{tabular*}{0.97\textwidth}{l@{\extracolsep{\fill}}r}
      \textit{\small#1} & \textit{\small #2} \\
    \end{tabular*}\vspace{-7pt}
}

\newcommand{\resumeProjectHeading}[2]{
    \item
    \begin{tabular*}{0.97\textwidth}{l@{\extracolsep{\fill}}r}
      \small#1 & #2 \\
    \end{tabular*}\vspace{-7pt}
}

\newcommand{\resumeSubItem}[1]{\resumeItem{#1}\vspace{-4pt}}

\renewcommand\labelitemii{$\vcenter{\hbox{\tiny$\bullet$}}$}

\newcommand{\resumeSubHeadingListStart}{\begin{itemize}[leftmargin=0.15in, label={}]}
\newcommand{\resumeSubHeadingListEnd}{\end{itemize}}
\newcommand{\resumeItemListStart}{\begin{itemize}}
\newcommand{\resumeItemListEnd}{\end{itemize}\vspace{-5pt}}

%-------------------------------------------
%%%%%%  RESUME STARTS HERE  %%%%%%%%%%%%%%%%%%%%%%%%%%%%


\begin{document}

%----------HEADING----------
% \begin{tabular*}{\textwidth}{l@{\extracolsep{\fill}}r}
%   \textbf{\href{http://sourabhbajaj.com/}{\Large Sourabh Bajaj}} & Email : \href{mailto:sourabh@sourabhbajaj.com}{sourabh@sourabhbajaj.com}\\
%   \href{http://sourabhbajaj.com/}{http://www.sourabhbajaj.com} & Mobile : +1-123-456-7890 \\
% \end{tabular*}

\begin{center}
	\textbf{\Huge \scshape Dylan Chen} \\ \vspace{1pt}
	\small +64 021-146-8719 $|$ \href{mailto:dche610@aucklanduni.ac.nz}{\underline{dche610@aucklanduni.ac.nz}} $|$
	\href{https://www.linkedin.com/in/dylan-chen-763b15200/}{\underline{linkedin.com/in/dylan-chen-763b15200}}
\end{center}


%
%-----------PROGRAMMING SKILLS-----------
\section{Technical Skills}
\begin{itemize}[leftmargin=0.15in, label={}]
	\small{\item{
		            \textbf{Languages}{: Python, R, SQL, Bash, JavaScript, HTML/CSS} \\
		            \textbf{Developer Tools}{: Git, Docker, Kubernetes, GitHub, GitHub Actions,  GitLab, Azure, AzureML, Azure DevOps, AWS, Linux, Snowflake, PowerBI, QGIS, CVAT} \\
		            \textbf{Frameworks}{: Tensorflow, Keras, PyTorch, Terraform, Pulumi, Flask, FastAPI, Spark} \\
		            \textbf{Certifications}{: Microsoft Certified: Azure Data Scientist Associate}
		      }}
\end{itemize}


%-----------EXPERIENCE-----------
\section{Experience}
\resumeSubHeadingListStart

\resumeSubheading
{Software Engineer}{Feb. 2023 -- Present}
{KPMG Lighthouse}{Auckland, New Zealand}
\resumeItemListStart
\resumeItem{Reduced batch inference time for a tax asset classifier by up to 80\% via integrating asychronous calls to OpenAI}
\resumeItem{Engineered Spark pipelines to serve dashboards for over 10,000 NZ Police staff using Azure Synapse Analytics}
\resumeItem{Reduced monthly spend on cloud infrastructure, saving up to \$4000 USD per month by migrating two retail analytics applications from AWS EC2 to Azure PaaS (App Service, Container Instance, Blob Storage)}
\resumeItem{Implemented data pipelines to classify risky driving behaviour for over 7 million St. John ambulance trips on IBM Cloud using Polars for data transformation and sci-kit learn for unsupervised machine learning}
\resumeItemListEnd

% -----------Multiple Positions Heading-----------
%    \resumeSubSubheading
%     {Software Engineer I}{Oct 2014 - Sep 2016}
%     \resumeItemListStart
%        \resumeItem{Apache Beam}
%          {Apache Beam is a unified model for defining both batch and streaming data-parallel processing pipelines}
%     \resumeItemListEnd
%    \resumeSubHeadingListEnd
%-------------------------------------------

\resumeSubheading
{Machine Learning Engineer Intern}{July 2022 -- Feb. 2023}
{Umajin}{Auckland, New Zealand}
\resumeItemListStart
\resumeItem{Reduced image labelling times by 50\% via deploying and configuring CVAT (image labelling platform) on AWS EC2 with Docker and implementing GPU-accelerated automatic annotation}
\resumeItem{Trained and deployed deep learning models for NTT and Starbucks which were presented by the CEO to more than 61,000 attendees at MWC 2022 using PyTorch, Keras and Flask}
\resumeItem{Automated the data preprocessing for over 20 different image datasets by developing scripts in Python}
\resumeItemListEnd

\resumeSubheading
{Data Scientist Intern}{Nov. 2021 -- Feb. 2022}
{Arup}{Auckland, New Zealand}
\resumeItemListStart
\resumeItem{Reduced time to validate simulation output by 7 days via implementing traffic volume benchmarks (traffic counter snapshots for different roads) for an agent-based simulation of transport in NZ using Python}
\resumeItem{Visualised and presented over 50 proposed transport infrastructure changes to research scientists in the development team by extracting data from AWS EFS and running geospatial queries with QGIS}
\resumeItem{Implemented scripts to process over 10 different GIS datasets from sources such as CoreLogic using geopandas}
\resumeItemListEnd

\resumeSubHeadingListEnd


%-----------PROJECTS-----------
\section{Projects}
\resumeSubHeadingListStart
\resumeProjectHeading
{\textbf{Object Detection for Trash Sorting} $|$ \emph{AzureML, MLFlow, PyTorch}}{Nov. 2023 -- Feb. 2024}
\resumeItemListStart
\resumeItem{Trained and deployed a computer vision model to detect 10 types of food waste using AzureML and MLFlow}
\resumeItem{Configured and deployed CVAT, which consists of 12 microservices, on Kubernetes (AKS) using a Helm chart}

\resumeItemListEnd
\resumeProjectHeading
{\textbf{Stroke Lesion Inpainting for 3D Brain MRI} $|$ \emph{Python, PyTorch, TorchIO}}{Feb. 2022 -- Oct. 2022}
\resumeItemListStart
\resumeItem{Trained neural networks to inpaint stroke lesions with healthy brain tissue using 500 brain scans in PyTorch}
\resumeItem{Applied brain image preprocessing techniques, reducing training time by 2 days using Python and TorchIO}

\resumeItemListEnd
\resumeProjectHeading
{\textbf{TakiWaehere Geospatial Hackathon} $|$ \emph{Python, geopandas, Matplotlib, Keras}}{Feb. 2021 -- April 2021}
\resumeItemListStart
\resumeItem{Researched the NZ Transport Agency Crash Analysis Systems dataset resulting in the identification of 3 features correlated with crashes (speed, weather, lighting) using pandas and Matplotlib}
\resumeItem{Gathered and presented research on GIS data in 4 days by showing a live demonstration using Jupyter Notebook}
\resumeItem{Developed scripts to preprocess 6 high-resolution satellite images of Auckland provided by Maxar in Python}
\resumeItemListEnd
\resumeSubHeadingListEnd




%-----------EDUCATION-----------
\section{Education}
\resumeSubHeadingListStart
\resumeSubheading
{University of Auckland}{Auckland, New Zealand}
{Bachelor of Engineering (Honours) in Engineering Science}{Feb. 2019 -- Nov. 2022}
\resumeItemListStart
\resumeItem{First Class Honours - GPA: 7.9/9.0 (A Average)}
\resumeItem{Dean's Honours List (2020-2021) - Top 5\% of Engineering Cohort}
\resumeItemListEnd
\resumeSubHeadingListEnd


%-------------------------------------------
\end{document}
